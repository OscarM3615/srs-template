\chapter{Requirements}

% This section specifies the software product's requirements. Specify all of the
% software requirements to a level of detail sufficient to enable designers to
% design a software system to satisfy those requirements, and to enable testers
% to test that the software system satisfies those requirements.

The specific requirements should:

\begin{itemize}
	\item Be uniquely identifiable.
	\item State the subject of the requirement (e.g., system, software, etc,.) and
		what shall be done.
	\item Optionally state the conditions and constraints, if any.
	\item Describe every input (stimulus) into the software system, every output
		(response) from the software system, and all functions performed by the
		software system in response to an input or support of an output.
	\item Be verifiable (e.g., the requirement realisation can be proven to the
		customer's satisfaction)
	\item Conform to agreed upon syntax, keywords, and terms.
\end{itemize}

\section{External Interfaces}

% This section defines all the inputs into and outputs requirements of the
% software system. Each interface defined may include the following content:

% Name of item
% Source of input or destination of output
% Valid range, accuracy, and/or tolerance
% Units of measure
% Timing
% Relationships to other inputs/outputs
% Screen formats/organisation
% Window formats/organisation
% Data formats
% Command formats
% End messages

\subsection{User Interfaces}

Define the software components for which a user interface is needed. Describe
the logical characteristics of each sentence between the software product and
the users. This may include sample screen images, any GUI standards or product
family style guides that are to be followed, screen layout constraints, standard
buttons and functions (e.g., help) that will appear on every screen, keyboard
shortcuts, error message display standards, and so on. Details of the user
interface design should be documented in a separate user interface
specification.

Could be further divided into Usability and Convenience requirements.

\subsection{Hardware Interfaces}

Describe the logical and physical characteristics of each interface between the
software product and the hardware components of the system. This may include the
supported device types, the nature of the data and control interactions between
the software and the hardware, and communication protocols to be used.

\subsection{Software Interfaces}

Describe the connections between this product and other specific software
components (name and version), including databases, operating systems, tools,
libraries, and integrated commercial components. Identify the data items or
messages coming into the system and going out and describe the purpose of each.
Describe the services needed and the nature of communications. Refer to
documents that describe detailed application programming interface protocols.
Identify data that will be shared across software components. If the data
sharing mechanism must be implemented in a specific way (for example, use of a
global data area in a multitasking operating system), specify this as an
implementation constraint.

\section{Functional Requirements}

% This section specifies the requirements of functional effects that the
% software-to-be is to have on its environment.

\section{Quality of Service}

% This section states additional, quality-related property requirements that the
% functional effects of the software should present.

\subsection{Performance}

If there are performance requirements for the product under various
circumstances, state them here and explain their rationale, to help the
developers understand the intent and make suitable design choices. Specify the
timing relationships for real time systems. Make such requirements as specific
as possible. You may need to state performance requirements for individual
functional requirements or features.

\subsection{Security}

Specify any requirements regarding security or privacy issues surrounding use of
the product or protection of the data used or created by the product. Define any
user identity authentication requirements. Refer to any external policies or
regulations containing security issues that affect the product. Define any
security or privacy certifications that must be satisfied.

\subsection{Reliability}

Specify the factors required to establish the required reliability of the
software system at time of delivery.

\subsection{Availability}

Specify the factors required to guarantee a defined availability level for the
entire system such as checkpoint, recovery, and restart.

\section{Compliance}

Specify the requirements derived from existing standards or regulations,
including:

\begin{itemize}
	\item Report format
	\item Data naming
	\item Accounting procedures
	\item Audit tracing
\end{itemize}

For example, this could specify the requirement for software to trace processing
activity. Such traces are needed for some applications to meet minimum
regulatory or financial standards. An audit trace requirement may, for example,
state that all changes to a payroll database shall be recorded in a trace file
with before and after values.

\section{Design and Implementation}

\subsection{Installation}

Constraints to ensure that the software-to-be will run smoothly on the target
implementation platform.

\subsection{Distribution}

Constraints on software components to fit the geographically distributed
structure of the host organisation, the distribution of data to be processed, or
the distribution of devices to be controlled.

\subsection{Maintainability}

Specify attributes of software that relate to the ease of maintenance of the
software itself. These may include requirements for certain modularity,
interfaces, or complexity limitation. Requirements should not be placed here
just because they are thought to be good design practices.

\subsection{Reusability}

\subsection{Portability}

Specify attributes of software that relate to the ease of porting the software
to other host machines and/or operating systems.

\subsection{Cost}

Specify monetary cost of the software product.

\subsection{Deadline}

Specify schedule for delivery of the software product.

\subsection{Proof of Concept}
