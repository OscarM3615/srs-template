\chapter{Product Overview}

% This section should describe the general factors that affect the product and
% its requirements. This sections does not state specific requirements. Instead,
% it provides a background for those requirements, which are defined in detail in
% Section 3, and makes them easier to understand.

\section{Product Perspective}

Describe the context and origin of the product being specified in this SRS. For
example, state whether this product is a follow-on member of a product family, a
replacement for certain existing systems, or a new, self-contained product. If
the SRS defines a component of a larger system, relate the requirements of the
larger system to the functionality of this software and identify interfaces
between the two. A simple diagram that shows the major components of the overall
system, subsystem interconnections, and external interfaces can be helpful.

\section{Product Functions}

Summarise the major functions the product must perform or must let the user
perform. Details will be provided in Section 3, so only a high level summary
(such as a bullet list) is needed here. Organise the functions to make them
understandable to any reader of the SRS. A picture of the major groups of
related requirements and how they relate, such as a top level data flow diagram
of object class diagram, is often effective.

\section{Product Constraints}

This subsection should provide a general description of any other items that
will limit the developer's options. These may include:

\begin{itemize}
	\item Interfaces tu users, other applications or hardware.
	\item Quality of service constraints.
	\item Standards compliance.
	\item Constraints around design or implementation.
\end{itemize}

\section{User Characteristics}

Identify the various user classes that you anticipate will use this product.
User classes may be differentiated based on frequency of use, subset of product
functions used, technical expertise, security or privilege levels, educational
level, or experience. Describe the pertinent characteristics of each user class.
Certain requirements may pertain only to certain user classes. Distinguish the
most important user classes for this product from those who are less important
to satisfy.

\section{Assumptions and Dependencies}

List any assumed factors (as opposed to known facts) that could affect the
requirements stated in the SRS. These could include third-party or commercial
components that you plan to use, issues around the development or operating
environment, or constraints. The project could be affected if these assumptions
are incorrect, are not shared, or change. Also identify any dependencies the
project has on external factors, such as software components that you intend to
reuse from another project, unless they are already documented elsewhere (for
example, in the vision and scope document or the project plan).

\section{Apportioning of Requirements}

Apportion the software requirements to software elements. For requirements that
will require implementation over multiple software elements, or when allocation
to a software element is initially undefined, this should be so stated. A cross
reference table by function and software element should be used to summarise the
apportioning.

Identify requirements that may be delayed until future versions of the system
(e.g., blocks and/or increments).
